\documentclass[11pt]{standalone}

\usepackage{helvet}

\usepackage{ifthen}
\usepackage{tikz} 
\usetikzlibrary{shapes.misc}
\usetikzlibrary{arrows,arrows.meta}
\usetikzlibrary{calc,intersections, patterns, math}
\usetikzlibrary{decorations.pathmorphing}
\usetikzlibrary{angles,quotes}

\definecolor{pfeil}{RGB}{168,167,167}
\definecolor{petrol}{RGB}{0, 118, 136}
\definecolor{darkgoldenrod}{RGB}{184, 134, 11}
\colorlet{petrol-lighter}{petrol!40}
\colorlet{darkgoldenrod-lighter}{darkgoldenrod!40} 


\pgfmathdeclarefunction{pa}{1}{\pgfmathparse{1/5000*(#1-7)^2*(#1+7)^2*(#1+8)}}

\pgfmathdeclarefunction{pb}{1}{\pgfmathparse{1/5000*(#1-9)*(#1-7)*(#1-3)*(#1+8)*(#1+9)}}

\pgfmathdeclarefunction{pc}{1}{\pgfmathparse{1/3000*(#1-9)*(#1-7)*(#1-6.5)*(#1-5.5)*(#1-3)*(#1-2.2)*(#1-0.5)*(#1+1)*(#1+1.5)+1}}

\pgfmathdeclarefunction{poly}{1}{\pgfmathparse{1/75*(#1+5)*(#1+2)*(#1-1)*(#1-6)}}
\pgfmathdeclarefunction{polya}{1}{\pgfmathparse{1/12*(#1+5)*(#1+2)*(#1-1)*(#1-1)}}
\pgfmathdeclarefunction{polyb}{1}{\pgfmathparse{1/20*(#1+5)*(#1+5)*(#1-1)*(#1-1)}}
\pgfmathdeclarefunction{polyc}{1}{\pgfmathparse{1/30*(#1+5)*(#1-1)*(#1-1)*(#1-1)}}
\pgfmathdeclarefunction{polyd}{1}{\pgfmathparse{1/30*(#1-1)*(#1-1)*(#1-1)*(#1-1)}}


\pgfmathdeclarefunction{pf}{1}{\pgfmathparse{1/7*(#1+7)*(#1+1)*(#1-3)}}
\pgfmathdeclarefunction{pg}{1}{\pgfmathparse{1/60*(#1+8)*(#1+2)*(#1-2)*(#1-6)}}
\pgfmathdeclarefunction{ph}{1}{\pgfmathparse{1/100*(#1+6)*(#1+4)*(#1-1)*(#1-4)*(#1-7)}}
\pgfmathdeclarefunction{pj}{1}{\pgfmathparse{1/2000*(#1+9)*(#1+5)*(#1+3)*#1*(#1-5)*(#1-8)}}

\begin{document}

\def\scl{0.25}

\begin{tikzpicture}[pfeil,>=stealth,scale=0.6]

	% \draw[thick, fill=petrol!20, draw=petrol-lighter, rounded corners=2ex, opacity=0.5] (0,0) rectangle ++ (1.5,3.5);
	% \draw[thick, fill=darkgoldenrod!20, draw=darkgoldenrod-lighter, rounded corners=2ex, opacity=0.5] (5,0) rectangle ++ (1.5,3.5);

	% Grid
	% \draw[step=1,lightgray] (-9.99,-13.99) grid (9.99,14.99);

	% Diagramm
	\draw[thick, -stealth] (-10,0) -- (10,0) node[below] {$x$};
	\draw[thick, -stealth] (0,-14) -- (0,15) node[left] {$y$};
	

	% Kurve
	\draw[ultra thick, petrol, samples=100, smooth, domain=-6:7] plot (\x,{ph(\x)});

	% Punkte
	\foreach \x in {-6,-4,1,4,7} %N
					\draw[fill,red] (\x,{ph(\x)}) circle(0.2);
					
					\foreach \x in {-5.1714,2.5504} %H
					\draw[fill,cyan] (\x,{ph(\x)}) circle(0.2);
					
					\foreach \x in {-1.6844,5.9054} %T
					\draw[fill,green] (\x,{ph(\x)}) circle(0.2);
					
					\foreach \x in {-3.8051,0.42457,4.5805} %W
					\draw[fill,orange] (\x,{ph(\x)}) circle(0.2);
			
			
				


\end{tikzpicture}

\end{document}
