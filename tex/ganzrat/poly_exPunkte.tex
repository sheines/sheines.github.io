\documentclass[11pt]{standalone}

\usepackage{helvet}

\usepackage{ifthen}
\usepackage{tikz} 
\usetikzlibrary{shapes.misc}
\usetikzlibrary{arrows,arrows.meta}
\usetikzlibrary{calc,intersections, patterns, math}
\usetikzlibrary{decorations.pathmorphing}
\usetikzlibrary{angles,quotes}

\definecolor{pfeil}{RGB}{168,167,167}
\definecolor{petrol}{RGB}{0, 118, 136}
\definecolor{darkgoldenrod}{RGB}{184, 134, 11}
\colorlet{petrol-lighter}{petrol!40}
\colorlet{darkgoldenrod-lighter}{darkgoldenrod!40} 


\pgfmathdeclarefunction{pa}{1}{\pgfmathparse{1/5000*(#1-7)^2*(#1+7)^2*(#1+8)}}

\pgfmathdeclarefunction{pb}{1}{\pgfmathparse{1/5000*(#1-9)*(#1-7)*(#1-3)*(#1+8)*(#1+9)}}

\pgfmathdeclarefunction{pc}{1}{\pgfmathparse{1/3000*(#1-9)*(#1-7)*(#1-6.5)*(#1-5.5)*(#1-3)*(#1-2.2)*(#1-0.5)*(#1+1)*(#1+1.5)+1}}

\pgfmathdeclarefunction{poly}{1}{\pgfmathparse{1/75*(#1+5)*(#1+2)*(#1-1)*(#1-6)}}
\pgfmathdeclarefunction{polya}{1}{\pgfmathparse{1/12*(#1+5)*(#1+2)*(#1-1)*(#1-1)}}
\pgfmathdeclarefunction{polyb}{1}{\pgfmathparse{1/20*(#1+5)*(#1+5)*(#1-1)*(#1-1)}}
\pgfmathdeclarefunction{polyc}{1}{\pgfmathparse{1/30*(#1+5)*(#1-1)*(#1-1)*(#1-1)}}
\pgfmathdeclarefunction{polyd}{1}{\pgfmathparse{1/30*(#1-1)*(#1-1)*(#1-1)*(#1-1)}}


\pgfmathdeclarefunction{pf}{1}{\pgfmathparse{1/7*(#1+7)*(#1+1)*(#1-3)}}
\pgfmathdeclarefunction{pg}{1}{\pgfmathparse{1/60*(#1+8)*(#1+2)*(#1-2)*(#1-6)}}
\pgfmathdeclarefunction{ph}{1}{\pgfmathparse{1/100*(#1+6)*(#1+4)*(#1-1)*(#1-4)*(#1-7)}}
\pgfmathdeclarefunction{pj}{1}{\pgfmathparse{1/2000*(#1+9)*(#1+5)*(#1+3)*#1*(#1-5)*(#1-8)}}

\begin{document}

\def\scl{0.25}

\begin{tikzpicture}[pfeil,>=stealth,scale=0.6]

	% \draw[thick, fill=petrol!20, draw=petrol-lighter, rounded corners=2ex, opacity=0.5] (0,0) rectangle ++ (1.5,3.5);
	% \draw[thick, fill=darkgoldenrod!20, draw=darkgoldenrod-lighter, rounded corners=2ex, opacity=0.5] (5,0) rectangle ++ (1.5,3.5);

	\draw[thick, -stealth] (-8,0) -- (8.5,0) node[below] {$x$};
	\draw[thick, -stealth] (0,-4.5) -- (0,5.5) node[left] {$y$};
						
	\draw[ultra thick, petrol, samples=100, smooth, domain=-6.0765:6.6569] plot (\x,{poly(\x)});

	%Nullstellen
	\draw[fill, red] (-5,{poly(-5)}) node[above right] {$x_\text{\tiny N}$} circle (0.1);
	\draw[fill, red] (-2,{poly(-2)}) node[above left] {$x_\text{\tiny N}$} circle (0.1);
	\draw[fill, red] (1,{poly(1)}) node[above right] {$x_\text{\tiny N}$} circle (0.1);
	\draw[fill, red] (6,{poly(6)}) node[below right] {$x_\text{\tiny N}$} circle (0.1);

	%Hochpunkte
	\draw[fill, cyan] (-0.42903,{poly(-0.42903)}) node[above] {$H$} circle (0.1);

	%Tiefpunkte
	\draw[fill, green] (-3.8305,{poly(-3.8305)}) node[below] {$T$} circle (0.1);
	\draw[fill, green] (4.2595,{poly(4.2595)}) node[above] {$T$} circle (0.1);

	%Wendepunkte
	\draw[fill, orange] (-2.3452,{poly(-2.3452)}) node[below right] {$W$} circle (0.1);
	\draw[fill, orange] (2.3452,{poly(2.3452)}) node[above right] {$W$} circle (0.1);

\end{tikzpicture}

\end{document}
