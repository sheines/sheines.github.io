\documentclass[12pt, parskip=half, fleqn, a4paper, twoside]{scrartcl}
\usepackage[ngerman]{babel}
\usepackage[utf8]{inputenc}
\usepackage{amsmath, amssymb, amsfonts}
\usepackage{lmodern, clock, graphicx, units, textcomp, multicol, booktabs}
\usepackage[Smaller]{cancel}
\usepackage{xlop}
\usepackage[left=1.5cm, right=1cm, top=1.7cm, bottom=2.3cm]{geometry}
\usepackage{icomma, paralist, array}

\usepackage{wasysym, pifont,xstring}      % Vierecke zum Ankreuzen (\Square) und bereits gekreuzte Vierecke (\XBox)

\usepackage[clock]{ifsym} % Uhrsymbol

\usepackage{stmaryrd}

\usepackage{polynom}

\renewcommand{\d}{\text{\, d}}
\newcommand{\e}{\text{e}}

%\setkomafont{section}{\sffamily\huge\centering}
%\setkomafont{subsection}{\normalfont\large\centering}

\newcounter{aufgabe}
\setcounter{aufgabe}{0}

\newcommand{\aufgabe}{\paragraph{Aufgabe \arabic{aufgabe}:}\stepcounter{aufgabe}}

\KOMAoptions{numbers=noenddot}

%Kopf und Fusszeilen definieren
\usepackage{scrlayer-scrpage}
\clearpairofpagestyles
\setkomafont{pageheadfoot}{\sffamily}
\setkomafont{pagination}{}

 
%\ohead[6. Oktober 2017]{6. Oktober 2017} 
%\ihead[Name, Vorname:]{Name,Vorname:} 
\cfoot[Seite \arabic{page}]{Seite \arabic{page}}
 
%\KOMAoptions{headsepline=true}\setheadsepline[\textwidth+1.5cm]{ 0.8pt}[\hspace*{-0.75cm}\centering]

\allowdisplaybreaks[1]

\let\v\vec

\renewcommand{\d}{\text\textdegree}

\renewcommand{\t}[1]{\text{\tiny{#1}}}
\newcommand{\sq}{\text{\raisebox{-0.3ex}{\huge\Square~}}}
\newcommand{\sa}{\raisebox{-1.5ex}{\rule{0pt}{0.7cm}}}
\newcommand{\sd}{{\Large\hspace*{-0.5ex}\textcolor{white}{I}}}
\newcommand{\strich}[1][1cm]{\raisebox{-1.5ex}{\rule{#1}{0.5pt}}} 

\newcolumntype{C}[1]{>{\centering\arraybackslash}m{#1}}

\newenvironment{aufgaben}{\begin{addmargin}[10pt]{0pt}\begin{enumerate}[a)]}{\end{enumerate}\end{addmargin}}


\usepackage{tikz}
\usetikzlibrary{shapes.misc}
\usetikzlibrary{arrows.meta}
\usetikzlibrary{angles,quotes}
%\usetikzlibrary{intersections}
\usetikzlibrary{calc}
\usetikzlibrary{decorations.text}

\usepackage{tikzsymbols}

\begin{document}


\title{Übungsheft Polynomdivision}
\date{12. Juni 2024}
\author{\strich[9cm]}

\maketitle
\vfill

\setcounter{tocdepth}{2}   
\thispagestyle{empty}
\tableofcontents

\vfill
\cleardoublepage

%\setcounter{section}{-1}

%\section{Selbsteinschätzung}

Erklärung:
\begin{itemize}
\item[\Ninja] Ich beherrsche diesen Aufgabentyp so sicher, dass ich ihn anderen erklären könnte.
\item[\Smiley] Ich bin sicher bei der Ausführung dieses Aufgabentyps.
\item[\Neutrey] Ich muss diesen Aufgabentyp noch üben.
\item[\Sadey] Ich brauche noch Hilfe bei der Lösung dieses Aufgabentyps
\end{itemize}

Die Seitenzahl gibt an, wo die entsprechenden Übungsaufgaben zu finden sind.

\begin{center}
\begin{tabular}{m{3.1cm}|C{6.2cm}|c|*{4}{|C{0.35cm}}|p{2.9cm}}
Aufgabentyp & Beispielaufgabe & Seite & \Ninja & \Smiley & \Neutrey & \Sadey & Häufigste Fehler \\ \hline \hline
Ich kann Summen addieren, subtrahieren und multiplizieren. & $(x-3)(x+5)=$ & \pageref{grch} &&&& \\ \hline

Ich kann mit den Potenzgesetzen umgehen. & $x^6:x^2=$ &\pageref{grch} &&&& \\ \hline

Ich kann sehr einfache Poly\-nom\-divi\-sionen durchführen. & $(x^2-4x+4):(x-2)=$ &\pageref{sech} &&&& \\ \hline

Ich kann einfache Poly\-nom\-divi\-sionen durchführen. & $(x^3-5x^2+2x-4):(x+3)=$ & \pageref{ech} &&&& \\ \hline

Ich kann komplexe Poly\-nom\-divis\-ionen durchführen. & $(x^3+3x^2-6x+2):(x^2+3x-1)=$ & \pageref{koch} &&&& \\ \hline

Ich kann Poly\-nom\-divi\-sion mit fehlenden Gliedern durchführen. & $(x^7-1):(x-1)=$ & \pageref{fgch} &&&& \\ \hline

Ich kann Poly\-nom\-divi\-sion mit Rest durchführen. & $(x^2+1):(x+5)=$ & \pageref{rech} &&&& \\ \hline

Ich kann Fehler in Poly\-nom\-divi\-sionen finden. & $3\cdot 5 = 12$ & \pageref{fech} &&&& \\ \hline

Ich bin sicher im Umgang mit allen oben genannten Aufgabentypen. & & \pageref{wech} &&&& 
\end{tabular}
\end{center}
%\cleardoublepage
\section{Grundlagen}\label{grch}

Lösungen auf Seite \pageref{grundlagen}.

Der Abschnitt Grundlagen sollte nur bearbeitet werden, wenn es Probleme beim Addieren, Subtrahieren und Multiplizieren von Summen oder bei der Anwendung der Potenzgesetze Probleme gibt.

\subsection{Terme addieren, subtrahieren und multiplizieren}

\begin{aufgaben}
\item $3 x^2-4 x-4 +  x-2 = $
\item $7 x^2+4 x-6 +  x^2-11 x+13 = $
\item $6 x^2-2 x-1 + \big(-4 x^2+7 x+5\big) = $ \\
\hrule
\item $3 x^3+4 x^2 - \big(2 x^3+3\big) =  $
\item $5 x^3+9 x+4 - \big(9 x-4\big) = $
\item $7 x^2-9 x+2 - \big(7 x^3+9 x^2-2 x\big) =$
\item $4 x^2-3 x+2 - \big(- x^2-3 x\big) = $
\item $8 x^2-6 x+2 - \big(3 x^2+7 x-4\big) = $ \\
\hrule
\item $\big(7 x+3\big) \cdot \big( x-3\big) = $ \\
\item $\big(9 x-1\big) \cdot \big(9 x+1\big) = $ \\
\item $\big( x^2+6\big) \cdot \big(-2 x^2+7\big) = $ \\
\item $\big( 1+x\big) \cdot \big( x^2-5 x-6\big) =  $ \\
\item $\big( x^2-3\big) \cdot \big(3 x^3-5 x^2+2 x-1\big) =$
\end{aufgaben}

\subsection{Potenzgesetze anwenden}

\begin{aufgaben}
\item $x^9\cdot x^5=$
\item $x^7\cdot x^2=$
\hrule
\item $x^7:x^5 = $
\item $x^{12}:x^{8}=$
\hrule
\item $\dfrac{x\cdot x\cdot x\cdot x\cdot x\cdot x\cdot x}{x\cdot x}=$
\end{aufgaben}
\cleardoublepage
\section{Sehr einfache Polynomdivision} \label{sech}

Lösung auf Seite \pageref{trivial}.

Dieses Kapitel sollte nur bearbeitet werden, wenn das Verfahren heute zum ersten Mal angewendet wird.

Lösen Sie die folgenden Quotienten mithilfe der Polynomdivision!

Beispiel:
\begin{center}
$\arraycolsep=1.4pt\newcommand{\pb}{\phantom{\big)}}
\begin{array}{*{25}{r}}
\big( x^2\pb&+\pb5 x\pb&+\pb6\big)&:&\big( x+3\big) =  x+2\\
-\big( x^2\pb&+\pb3 x\big) \\\cline{1-2}&
2 x\pb&+\pb6\pb\\&
-\big(2 x\pb&+\pb6\big) \\\cline{2-3}&& 0\pb
\end{array}$
\end{center}

\begin{aufgaben}
\item $\arraycolsep=1.4pt\newcommand{\pb}{\phantom{\big)}}
\begin{array}[t]{*{25}{r}}
\big( x^2\pb&+\pb2 x\pb&-\pb15\big)&:&\big( x+5\big) = 
\end{array}$
\vfill
\item $\arraycolsep=1.4pt\newcommand{\pb}{\phantom{\big)}}
\begin{array}[t]{*{25}{r}}
\big( x^2\pb&-\pb8 x\pb&+\pb7\big)&:&\big( x-7\big) = 
\end{array}$
\vfill
\item $\arraycolsep=1.4pt\newcommand{\pb}{\phantom{\big)}}
\begin{array}[t]{*{25}{r}}
\big( x^2\pb&+\pb12 x\pb&+\pb27\big)&:&\big( x+9\big) =  
\end{array}$
\vfill
\item $\arraycolsep=1.4pt\newcommand{\pb}{\phantom{\big)}}
\begin{array}[t]{*{25}{r}}
\big(2 x^2\pb&+\pb3 x\pb&-\pb20\big)&:&\big( x+4\big) = 
\end{array}$
\vfill
\item $\arraycolsep=1.4pt\newcommand{\pb}{\phantom{\big)}}
\begin{array}[t]{*{25}{r}}
\big(12 x^2\pb&-\pb18 x\pb&-\pb12\big)&:&\big(4 x+2\big) = 
\end{array}$
\vfill
\end{aufgaben}
\cleardoublepage
\section{Einfache Polynomdivision}\label{ech}

Lösung auf Seite \pageref{einfach}.

Lösen Sie die folgenden Quotienten mithilfe der Polynomdivision!

\begin{aufgaben}
\item $\arraycolsep=1.4pt\newcommand{\pb}{\phantom{\big)}}
\begin{array}[t]{*{25}{r}}
\big( x^3\pb&-\pb x^2\pb&-\pb24 x\pb&-\pb36\big)&:&\big( x+2\big) =
\end{array}$
\vfill
\item $\arraycolsep=1.4pt\newcommand{\pb}{\phantom{\big)}}
\begin{array}[t]{*{25}{r}}
\big( x^3\pb&-\pb5 x^2\pb&-\pb18 x\pb&+\pb72\big)&:&\big( x-6\big) = 
\end{array}$
\vfill
\item $\arraycolsep=1.4pt\newcommand{\pb}{\phantom{\big)}}
\begin{array}[t]{*{25}{r}}
\big(24 x^3\pb&+\pb52 x^2\pb&+\pb16 x\pb&-\pb12\big)&:&\big(3 x-1\big) = 
\end{array}$
\vfill
\item $\arraycolsep=1.4pt\newcommand{\pb}{\phantom{\big)}}
\begin{array}[t]{*{25}{r}}
\big(24 x^5\pb&+\pb76 x^4\pb&+\pb20 x^3\pb&-\pb100 x^2\pb&-\pb44 x\pb&+\pb24\big)&:&\big(2 x+3\big) = 
\end{array}$
\vfill
\end{aufgaben}
\vfill
\cleardoublepage
\section{Komplexe Polynomdivision}\label{koch}

Lösung auf Seite \pageref{schwerer}.

Lösen Sie die folgenden Quotienten mithilfe der Polynomdivision!

\begin{aufgaben}
\item $\arraycolsep=1.4pt\newcommand{\pb}{\phantom{\big)}}
\begin{array}[t]{*{25}{r}}
\big( x^3\pb&-\pb8 x^2\pb&+\pb17 x\pb&-\pb6\big)&:&\big( x^2-5 x+2\big) = 
\end{array}$
\vfill
\item $\arraycolsep=1.4pt\newcommand{\pb}{\phantom{\big)}}
\begin{array}[t]{*{25}{r}}
\big( x^4\pb&+\pb8 x^3\pb&+\pb17 x^2\pb&+\pb6 x\pb&+\pb8\big)&:&\big( x^3+4 x^2+ x+2\big) =
\end{array}$
\vfill
\item $\arraycolsep=1.4pt\newcommand{\pb}{\phantom{\big)}}
\begin{array}[t]{*{25}{r}}
\big(6 x^4\pb&-\pb13 x^3\pb&+\pb9 x^2\pb&+\pb13 x\pb&-\pb10\big)&:&\big(2 x^3-3 x^2+ x+5\big) =
\end{array}$
\vfill
\item $\arraycolsep=1.4pt\newcommand{\pb}{\phantom{\big)}}
\begin{array}[t]{*{25}{r}}
\big(4 x^5\pb&-\pb14 x^4\pb&+\pb16 x^3\pb&+\pb3 x^2\pb&-\pb19 x\pb&+\pb5\big)&:&\big(2 x^2-4 x+1\big) = 
\end{array}$
\vfill
\item $\arraycolsep=1.4pt\newcommand{\pb}{\phantom{\big)}}
\begin{array}[t]{*{25}{r}}
\big(20 x^6\pb&+\pb2 x^5\pb&-\pb31 x^4\pb&+\pb4 x^3\pb&+\pb7 x^2\pb&-\pb5 x\pb&+\pb3\big)&:&\big(4 x^2+2 x-3\big) = 
\end{array}$
\vfill
\end{aufgaben}
\vfill
\cleardoublepage
\section{Polynomdivision mit fehlenden Gliedern}\label{fgch}

Lösung auf Seite \pageref{fehlend}.

Lösen Sie die folgenden Quotienten mithilfe der Polynomdivision!

\begin{aufgaben}
\item $\arraycolsep=1.4pt\newcommand{\pb}{\phantom{\big)}}
\begin{array}[t]{*{25}{r}}
\big( x^3\pb&&&-\pb1\big)&:&\big( x-1\big) =  
\end{array}$
\vfill
\item $\arraycolsep=1.4pt\newcommand{\pb}{\phantom{\big)}}
\begin{array}[t]{*{25}{r}}
\big( x^9\pb&&&-\pb1\big)&:&\big( x^3-1\big) = 
\end{array}$
\vfill
\item $\arraycolsep=1.4pt\newcommand{\pb}{\phantom{\big)}}
\begin{array}[t]{*{25}{r}}
\big( x^4\pb&-\pb7 x^3\pb&-\pb5 x\pb&+\pb35\big)&:&\big( x-7\big) =  
\end{array}$
\vfill
\item $\arraycolsep=1.4pt\newcommand{\pb}{\phantom{\big)}}
\begin{array}[t]{*{25}{r}}
\big( x^5\pb&-\pb2 x^3\pb&-\pb8 x\big)&:&\big( x^2+2\big) =  
\end{array}$
\vfill
\cleardoublepage
\item $\arraycolsep=1.4pt\newcommand{\pb}{\phantom{\big)}}
\begin{array}[t]{*{25}{r}}
\big( x^7\pb&&+\pb2 x^4\pb&&+\pb2 x^2\pb&-\pb x\pb&+\pb2\big)&:&\big( x^4+ x^2+1\big) = 
\end{array}$
\vfill
\item $\arraycolsep=1.4pt\newcommand{\pb}{\phantom{\big)}}
\begin{array}[t]{*{25}{r}}
\big(6 x^6\pb&+\pb6 x^4\pb&+\pb4 x^3\pb&-\pb10 x^2\pb&+\pb4 x\pb&-\pb10\big)&:&\big(2 x^2+2\big) = 
\end{array}$
\vfill
\item $\arraycolsep=1.4pt\newcommand{\pb}{\phantom{\big)}}
\begin{array}[t]{*{25}{r}}
\big( x^9\pb&-\pb512\big)&:&\big( x-2\big) = 
\end{array}$
\vfill
\end{aufgaben}
\vfill
\cleardoublepage
\section{Polynomdivision mit Rest}\label{rech}

Lösung auf Seite \pageref{rest}.

Lösen Sie die folgenden Quotienten mithilfe der Polynomdivision.

\begin{aufgaben}
\item $\arraycolsep=1.4pt\newcommand{\pb}{\phantom{\big)}}
\begin{array}[t]{*{25}{r}}
\big( x^2\pb&-\pb4 x\pb&+\pb3\big)&:&\big( x+5\big) = 
\end{array}$
\vfill
\item $\arraycolsep=1.4pt\newcommand{\pb}{\phantom{\big)}}
\begin{array}[t]{*{25}{r}}
\big( x^2\pb&-\pb7 x\pb&-\pb1\big)&:&\big( x-2\big) = 
\end{array}$
\vfill
\item $\arraycolsep=1.4pt\newcommand{\pb}{\phantom{\big)}}
\begin{array}[t]{*{25}{r}}
\big( x^2\pb&+\pb10 x\pb&+\pb25\big)&:&\big( x-5\big) = 
\end{array}$
\vfill
\item $\arraycolsep=1.4pt\newcommand{\pb}{\phantom{\big)}}
\begin{array}[t]{*{25}{r}}
\big( x^4\pb&+\pb2 x^3\pb&+\pb6 x^2\pb&&+\pb1\big)&:&\big( x+3\big) =  
\end{array}$
\vfill
\item $\arraycolsep=1.4pt\newcommand{\pb}{\phantom{\big)}}
\begin{array}[t]{*{25}{r}}
\big( x^3\pb&-\pb4 x^2\pb&+\pb2 x\pb&-\pb5\big)&:&\big( x^2+3\big) = 
\end{array}$
\vfill
\item $\arraycolsep=1.4pt\newcommand{\pb}{\phantom{\big)}}
\begin{array}[t]{*{25}{r}}
\big( x^5\pb&-\pb x^4\pb&+\pb x^3\pb&-\pb x^2\pb&+\pb x\pb&-\pb1\big)&:&\big( x^3- x^2+1\big) = 
\end{array}$
\vfill
\end{aufgaben}
\cleardoublepage
\section{Polynomdivision mit Fehlern}\label{fech}

Lösung auf Seite \pageref{fehler}.

Durch Probieren wurde eine Lösung der folgenden Polynome gefunden. Diese wurde nun per Polynomdivision abgespalten. Dabei sind jedoch einige Fehler aufgetreten. 

Finden und korrigieren Sie diese!

\begin{aufgaben}
\item $\arraycolsep=1.4pt\newcommand{\pb}{\phantom{\big)}}
\begin{array}[t]{*{25}{r}}
\big( x^2\pb&-\pb2 x\pb&-\pb3\big)&:&\big( x-3\big) =  x-5 + \dfrac{-18}{x-3}\\
-\big( x^2\pb&-\pb3 x\big) \\\cline{1-2}&
 -5x\pb&-\pb3\pb\\&
-\big( -5x\pb&+\pb15\big) \\\cline{2-3}
&& -18\pb
\end{array}$
\vfill
\item $\arraycolsep=1.4pt\newcommand{\pb}{\phantom{\big)}}
\begin{array}[t]{*{25}{r}}
\big( x^2\pb&+\pb5 x\pb&+\pb6\big)&:&\big( x+2\big) =  x+7+ \dfrac{20}{x+2}\\
-\big( x^2\pb&+\pb2 x\big) \\\cline{1-2}&
7 x\pb&+\pb6\pb\\&
-\big(7 x\pb&+\pb14\big) \\\cline{2-3}&&
	20\pb
\end{array}$
\vfill
\item $\arraycolsep=1.4pt\newcommand{\pb}{\phantom{\big)}}
\begin{array}[t]{*{25}{r}}
\big( x^3\pb&-\pb10 x^2\pb&+\pb17 x\pb&-\pb8\big)&:&\big( x-8\big) =  x^2+2 x+1\\
-\big( x^3\pb&-\pb8 x^2\big) \\\cline{1-2}&
\pb 2 x^2\pb&+\pb17 x\pb&-\pb8\pb\\&
-\big(\pb2 x^2\pb&-\pb16 x\big) \\\cline{2-3}&&
 x\pb&-\pb8\pb\\&&
-\big( \pb x\pb&-\pb8\big) \\\cline{3-4} &&& 0\pb
\end{array}$
\vfill
\item $\arraycolsep=1.4pt\newcommand{\pb}{\phantom{\big)}}
\begin{array}[t]{*{25}{r}}
\big( x^3\pb&-\pb6 x^2\pb&+\pb4 x\pb&+\pb8\big)&:&\big( x-2\big) = x-6 + \dfrac{6x-4}{x-2}\\
-\big( x^3\pb&&-\pb2 x\big) \\\cline{1-3}&
-\pb6 x^2\pb&+\pb6 x\pb&+\pb8\\&
-\big(-\pb6 x^2\pb&&+\pb12 \big) \\\cline{2-4}&&
\pb6 x\pb&-\pb4\pb
\end{array}$
\vfill
\item $\arraycolsep=1.4pt\newcommand{\pb}{\phantom{\big)}}
\begin{array}[t]{*{25}{r}}
\big( x^4\pb&+\pb5 x^3\pb&-\pb52 x^2\pb&-\pb356 x\pb&-\pb480\big)&:&\big( x+5\big) =  x^3\\
-\big( x^4\pb&+\pb5 x^3\big) \\\cline{1-2}& 0\pb
\end{array}$
\vfill
\end{aufgaben}

\cleardoublepage
\section{Weiterführende Aufgaben}\label{wech}

Lösung auf Seite \pageref{weiter}.

\begin{aufgaben}
\item Beim Bestimmen der Lösungen der Gleichung $0= x^3 -3 x^2 +4x -12$ wurde $x_1=3$ als Lösung gefunden. Nach Abspaltung des Linearfaktors $(x-3)$ blieb $0= x^2+4$ übrig. 

Erklären Sie, was dieses Ergebnis für die Anzahl der Lösungen bedeutet!
\vspace{4cm}

\item Berechnen Sie alle Nullstellen der Funktion $y=f(x)=x^5 - 4 x^4 - 15 x^3 + 50 x^2 + 64 x - 96$ und stellen Sie anschließend deren Linearfaktorzerlegung auf!
\vfill

\cleardoublepage

\item Erläutern Sie Unterschiede und Gemeinsamkeiten zwischen den beiden dargestellten Polynomdivisionen!

\begin{multicols}{2}
\arraycolsep=1.4pt\newcommand{\pb}{\phantom{\big)}}
\begin{array}[t]{*{25}{r}}
\big( x^2\pb&-\pb4 x\pb&-\pb21\big)&:&\big( x-7\big) =  x+3\\
-\big( x^2\pb&-\pb7 x\big) \\\cline{1-2}&
3 x\pb&-\pb21\pb\\&
-\big(3 x\pb&-\pb21\big) \\\cline{2-3}&&0\pb
\end{array}

\columnbreak

\polylongdiv[style=C,div=:]{x^2-4x-21}{x-7}
\end{multicols}
Begründen Sie, warum beide Schreibweisen mathematisch korrekt sind!
\vspace*{5cm}

\item Bestimmen Sie den Parameter $a$ so, dass die folgenden Gleichungen stimmen! Erklären Sie Ihr Vorgehen! Können Sie mehr als eine Vorgehensweise beschreiben?
\begin{enumerate}[aa)]
\item $(x^2+6x+8):(x+a) = x+4$
\vfill
\item $(x^2-3x-10):(x+a)=x-5$
\vfill
\item $(x^2-4x+16):(x+a)=x-4$
\vfill
\item $(x^3 + 8 x^2 - 69 x - 252):(x+a)=x^2-4x-21$
\vfill
\end{enumerate}

\cleardoublepage

\item Schreiben Sie einem/r Freund/in einen Brief in dem Sie ihm/ihr mit eigenen Worten das Verfahren der Polynomdivision erklären!
\vspace*{7cm}
\item Diskutieren Sie Vor- und Nachteile der beiden bekannten Schreibweisen für ganzrationale Funktionen:
\begin{itemize}
\item Allgemeine Form: $y = f(x)=3 x^4 - 9 x^3 - 24 x^2 + 36 x + 48$
\item Linearfaktorzerlegung: $y = f(x)=3(x+1)(x-2)(x+2)(x-4)$
\end{itemize}
\vfill
\item Die nachfolgend dargestellte Funktionsgleichung soll die gleiche Funktion wie in Aufgabe f) beschreiben.
\begin{align*}
y &= f(x)=(((3x-9)x-24)x+36)x+48
\end{align*}
Zeigen Sie, dass diese Schreibweise in die allgemeine Form überführbar ist!
\vspace*{5cm}

Notieren Sie wofür diese Funktionsschreibweise in der Praxis genutzt werden könnte!

\vspace*{3cm}
\end{aufgaben}


\appendix
\setcounter{section}{11}
\chead{Lösungen}
\section{Lösungen}

\subsection{Grundlagen}\label{grundlagen}
\chead{Lösungen -- Grundlagen}

\subsubsection{Terme addieren, subtrahieren und multiplizieren}

\begin{aufgaben}
\item $3 x^2-4 x-4 +  x-2 = 3 x^2-3 x-6$
\item $7 x^2+4 x-6 +  x^2-11 x+13 = 8 x^2-7 x+7$
\item $6 x^2-2 x-1 + \big(-4 x^2+7 x+5\big) = 2 x^2+5 x+4$ \\
\hrule
\item $3 x^3+4 x^2 - \big(2 x^3+3\big) =  x^3+4 x^2-3$
\item $5 x^3+9 x+4 - \big(9 x-4\big) = 5 x^3+8$
\item $7 x^2-9 x+2 - \big(7 x^3+9 x^2-2 x\big) = -7 x^3-2 x^2-7 x+2$
\item $4 x^2-3 x+2 - \big(- x^2-3 x\big) = 5 x^2+2$
\item $8 x^2-6 x+2 - \big(3 x^2+7 x-4\big) = 5 x^2-13 x+6$ \\
\hrule
\item $\big(7 x+3\big) \cdot \big( x-3\big) = 7 x^2-18 x-9$ \\
\item $\big(9 x-1\big) \cdot \big(9 x+1\big) = 81 x^2-1$ \\
\item $\big( x^2+6\big) \cdot \big(-2 x^2+7\big) = -2 x^4-5 x^2+42$ \\
\item $\big( 1+x\big) \cdot \big( x^2-5 x-6\big) =  x^3-4 x^2-11 x-6$ \\
\item $\big( x^2-3\big) \cdot \big(3 x^3-5 x^2+2 x-1\big) = 3 x^5-5 x^4-7 x^3+14 x^2-6 x+3$
\end{aufgaben}

\subsubsection{Potenzgesetze anwenden}

\begin{aufgaben}
\item $x^9\cdot x^5=x^{15}$
\item $x^7\cdot x^2=x^9$
\hrule
\item $x^7:x^5 = x^2$
\item $x^{12}:x^{8}=x^4$
\hrule
\item $\dfrac{x\cdot x\cdot x\cdot x\cdot x\cdot x\cdot x}{x\cdot x}=x^5$
\end{aufgaben}
\subsection{Sehr einfache Polynomdivision}\label{trivial}
\chead{Lösungen - Sehr einfache Polynomdivision}

\begin{aufgaben}
\item $\arraycolsep=1.4pt\newcommand{\pb}{\phantom{\big)}}
\begin{array}[t]{*{25}{r}}
\big( x^2\pb&+\pb2 x\pb&-\pb15\big)&:&\big( x+5\big) =  x-3\\
-\big( x^2\pb&+\pb5 x\big) \\\cline{1-2}&
-\pb3 x\pb&-\pb15\pb\\&
-\big(-\pb3 x\pb&-\pb15\big) \\\cline{2-3}&& 0\pb
\end{array}$
\item $\arraycolsep=1.4pt\newcommand{\pb}{\phantom{\big)}}
\begin{array}[t]{*{25}{r}}
\big( x^2\pb&-\pb8 x\pb&+\pb7\big)&:&\big( x-7\big) =  x-1\\
-\big( x^2\pb&-\pb7 x\big) \\\cline{1-2}&
-\pb x\pb&+\pb7\pb\\&
-\big(-\pb x\pb&+\pb7\big) \\\cline{2-3}&& 0 \pb
\end{array}$
\item $\arraycolsep=1.4pt\newcommand{\pb}{\phantom{\big)}}
\begin{array}[t]{*{25}{r}}
\big( x^2\pb&+\pb12 x\pb&+\pb27\big)&:&\big( x+9\big) =  x+3\\
-\big( x^2\pb&+\pb9 x\big) \\\cline{1-2}&
3 x\pb&+\pb27\pb\\&
-\big(3 x\pb&+\pb27\big) \\\cline{2-3}&& 0\pb
\end{array}$
\item $\arraycolsep=1.4pt\newcommand{\pb}{\phantom{\big)}}
\begin{array}[t]{*{25}{r}}
\big(2 x^2\pb&+\pb3 x\pb&-\pb20\big)&:&\big( x+4\big) = 2 x-5\\
-\big(2 x^2\pb&+\pb8 x\big) \\\cline{1-2}&
-\pb5 x\pb&-\pb20\pb\\&
-\big(-\pb5 x\pb&-\pb20\big) \\\cline{2-3}&& 0\pb
\end{array}$
\item $\arraycolsep=1.4pt\newcommand{\pb}{\phantom{\big)}}
\begin{array}[t]{*{25}{r}}
\big(12 x^2\pb&-\pb18 x\pb&-\pb12\big)&:&\big(4 x+2\big) = 3 x-6\\
-\big(12 x^2\pb&+\pb6 x\big) \\\cline{1-2}&
-\pb24 x\pb&-\pb12\pb\\&
-\big(-\pb24 x\pb&-\pb12\big) \\\cline{2-3}&& 0\pb
\end{array}$
\end{aufgaben}
\subsection{Einfache Polynomdivision}\label{einfach}
\chead{Lösungen -- Einfache Polynomdivision}

\begin{aufgaben}
\item $\arraycolsep=1.4pt\newcommand{\pb}{\phantom{\big)}}
\begin{array}[t]{*{25}{r}}
\big( x^3\pb&-\pb x^2\pb&-\pb24 x\pb&-\pb36\big)&:&\big( x+2\big) =  x^2-3 x-18\\
-\big( x^3\pb&+\pb2 x^2\big) \\\cline{1-2}&
-\pb3 x^2\pb&-\pb24 x\pb\\&
-\big(-\pb3 x^2\pb&-\pb6 x\big) \\\cline{2-3}&&
-\pb18 x\pb&-\pb36\pb\\&&
-\big(-\pb18 x\pb&-\pb36\big) \\\cline{3-4} &&& 0\pb
\end{array}$
\item $\arraycolsep=1.4pt\newcommand{\pb}{\phantom{\big)}}
\begin{array}[t]{*{25}{r}}
\big( x^3\pb&-\pb5 x^2\pb&-\pb18 x\pb&+\pb72\big)&:&\big( x-6\big) =  x^2+ x-12\\
-\big( x^3\pb&-\pb6 x^2\big) \\\cline{1-2}&
 x^2\pb&-\pb18 x\pb\pb\\&
-\big( x^2\pb&-\pb6 x\big) \\\cline{2-3}&&
-\pb12 x\pb&+\pb72\pb\\&&
-\big(-\pb12 x\pb&+\pb72\big) \\\cline{3-4}&&& 0\pb
\end{array}$
\item $\arraycolsep=1.4pt\newcommand{\pb}{\phantom{\big)}}
\begin{array}[t]{*{25}{r}}
\big(24 x^3\pb&+\pb52 x^2\pb&+\pb16 x\pb&-\pb12\big)&:&\big(3 x-1\big) = 8 x^2+20 x+12\\
-\big(24 x^3\pb&-\pb8 x^2\big) \\\cline{1-2}&
60 x^2\pb&+\pb16 x\pb\\&
-\big(60 x^2\pb&-\pb20 x\big) \\\cline{2-3}&&
36 x\pb&-\pb12\pb\\&&
-\big(36 x\pb&-\pb12\big) \\\cline{3-4}&&&0\pb
\end{array}$
\item $\arraycolsep=1.4pt\newcommand{\pb}{\phantom{\big)}}
\begin{array}[t]{*{25}{r}}
\big(24 x^5\pb&+\pb76 x^4\pb&+\pb20 x^3\pb&-\pb100 x^2\pb&-\pb44 x\pb&+\pb24\big)&:&\big(2 x+3\big) = \\
-\big(24 x^5\pb&+\pb36 x^4\big) \\\cline{1-2}&
40 x^4\pb&+\pb20 x^3\pb\\&
-\big(40 x^4\pb&+\pb60 x^3\big) \\\cline{2-3}&&
-\pb40 x^3\pb&-\pb100 x^2\pb\\&&
-\big(-\pb40 x^3\pb&-\pb60 x^2\big) \\\cline{3-4}&&&
-\pb40 x^2\pb&-\pb44 x\pb\\&&&
-\big(-\pb40 x^2\pb&-\pb60 x\big) \\\cline{4-5}&&&&
16 x\pb&+\pb24\pb\\&&&&
-\big(16 x\pb&+\pb24\big) \\\cline{5-6} \multicolumn{4}{l}{=12 x^4+20 x^3-20 x^2-20 x+8}&& 0\pb
\end{array}$
\end{aufgaben}
\subsection{Komplexe Polynomdivision}\label{schwerer}
\chead{Lösungen -- Komplexe Polynomdivision}

\begin{aufgaben}
\item $\arraycolsep=1.4pt\newcommand{\pb}{\phantom{\big)}}
\begin{array}[t]{*{25}{r}}
\big( x^3\pb&-\pb8 x^2\pb&+\pb17 x\pb&-\pb6\big)&:&\big( x^2-5 x+2\big) =  x-3\\
-\big( x^3\pb&-\pb5 x^2\pb&+\pb2 x\big) \\\cline{1-3}&
-\pb3 x^2\pb&+\pb15 x\pb&-\pb6\pb\\&
-\big(-\pb3 x^2\pb&+\pb15 x\pb&-\pb6\big) \\\cline{2-4}&&& 0\pb
\end{array}$
\item $\arraycolsep=1.4pt\newcommand{\pb}{\phantom{\big)}}
\begin{array}[t]{*{25}{r}}
\big( x^4\pb&+\pb8 x^3\pb&+\pb17 x^2\pb&+\pb6 x\pb&+\pb8\big)&:&\big( x^3+4 x^2+ x+2\big) =  x+4\\
-\big( x^4\pb&+\pb4 x^3\pb&+\pb x^2\pb&+\pb2 x\big) \\\cline{1-4}&
4 x^3\pb&+\pb16 x^2\pb&+\pb4 x\pb&+\pb8\pb\\&
-\big(4 x^3\pb&+\pb16 x^2\pb&+\pb4 x\pb&+\pb8\big) \\\cline{2-5} &&&& 0\pb
\end{array}$
\item $\arraycolsep=1.4pt\newcommand{\pb}{\phantom{\big)}}
\begin{array}[t]{*{25}{r}}
\big(6 x^4\pb&-\pb13 x^3\pb&+\pb9 x^2\pb&+\pb13 x\pb&-\pb10\big)&:&\big(2 x^3-3 x^2+ x+5\big) = 3 x-2\\
-\big(6 x^4\pb&-\pb9 x^3\pb&+\pb3 x^2\pb&+\pb15 x\big) \\\cline{1-4}&
-\pb4 x^3\pb&+\pb6 x^2\pb&-\pb2 x\pb&-\pb10\pb\\&
-\big(-\pb4 x^3\pb&+\pb6 x^2\pb&-\pb2 x\pb&-\pb10\big) \\\cline{2-5}&&&& 0\pb
\end{array}$
\item $\arraycolsep=1.4pt\newcommand{\pb}{\phantom{\big)}}
\begin{array}[t]{*{25}{r}}
\big(4 x^5\pb&-\pb14 x^4\pb&+\pb16 x^3\pb&+\pb3 x^2\pb&-\pb19 x\pb&+\pb5\big)&:&\big(2 x^2-4 x+1\big) = 2 x^3-3 x^2+ x+5\\
-\big(4 x^5\pb&-\pb8 x^4\pb&+\pb2 x^3\big) \\\cline{1-3}&
-\pb6 x^4\pb&+\pb14 x^3\pb&+\pb3 x^2\pb\\&
-\big(-\pb6 x^4\pb&+\pb12 x^3\pb&-\pb3 x^2\big) \\\cline{2-4}&&
2 x^3\pb&+\pb6 x^2\pb&-\pb19 x\pb\\&&
-\big(2 x^3\pb&-\pb4 x^2\pb&+\pb x\big) \\\cline{3-5}&&&
10 x^2\pb&-\pb20 x\pb&+\pb5\pb\\&&&
-\big(10 x^2\pb&-\pb20 x\pb&+\pb5\big) \\\cline{4-6}&&&&& 0\pb
\end{array}$
\item $\arraycolsep=1.4pt\newcommand{\pb}{\phantom{\big)}}
\begin{array}[t]{*{25}{r}}
\big(20 x^6\pb&+\pb2 x^5\pb&-\pb31 x^4\pb&+\pb4 x^3\pb&+\pb7 x^2\pb&-\pb5 x\pb&+\pb3\big)&:&\big(4 x^2+2 x-3\big) = \\
-\big(20 x^6\pb&+\pb10 x^5\pb&-\pb15 x^4\big) \\\cline{1-3}&
-\pb8 x^5\pb&-\pb16 x^4\pb&+\pb4 x^3\pb\\&
-\big(-\pb8 x^5\pb&-\pb4 x^4\pb&+\pb6 x^3\big) \\\cline{2-4}&&
-\pb12 x^4\pb&-\pb2 x^3\pb&+\pb7 x^2\pb\\&&
-\big(-\pb12 x^4\pb&-\pb6 x^3\pb&+\pb9 x^2\big) \\\cline{3-5}&&&
4 x^3\pb&-\pb2 x^2\pb&-\pb5 x\pb\\&&&
-\big(4 x^3\pb&+\pb2 x^2\pb&-\pb3 x\big) \\\cline{4-6}&&&&
-\pb4 x^2\pb&-\pb2 x\pb&+\pb3\pb\\&&&&
-\big(-\pb4 x^2\pb&-\pb2 x\pb&+\pb3\big) \\\cline{5-7}\multicolumn{4}{l}{=5 x^4-2 x^3-3 x^2+ x-1}&&& 0\pb 
\end{array}$
\end{aufgaben}
\subsection{Polynomdivision mit fehlenden Gliedern} \label{fehlend}
\chead{Lösungen -- Polynomdivision mit fehlenden Gliedern}

\begin{aufgaben}
\item $\arraycolsep=1.4pt\newcommand{\pb}{\phantom{\big)}}
\begin{array}[t]{*{25}{r}}
\big( x^3\pb&&&-\pb1\big)&:&\big( x-1\big) =  x^2+ x+1\\
-\big( x^3\pb&-\pb x^2\big) \\\cline{1-2}&
 x^2\pb&&-\pb1\pb\\&
-\big( x^2\pb&-\pb x\big) \\\cline{2-3}&&
 x\pb&-\pb1\pb\\&&
-\big( x\pb&-\pb1\big) \\\cline{3-4}&&&0\pb
\end{array}$
\item $\arraycolsep=1.4pt\newcommand{\pb}{\phantom{\big)}}
\begin{array}[t]{*{25}{r}}
\big( x^9\pb&&&-\pb1\big)&:&\big( x^3-1\big) =  x^6+ x^3+1\\
-\big( x^9\pb&-\pb x^6\big) \\\cline{1-2}&
 x^6\pb&&-\pb1\pb\\&
-\big( x^6\pb&-\pb x^3\big) \\\cline{2-3}&&
 x^3\pb&-\pb1\pb\\&&
-\big( x^3\pb&-\pb1\big) \\\cline{3-4}&&&0\pb
\end{array}$
\item $\arraycolsep=1.4pt\newcommand{\pb}{\phantom{\big)}}
\begin{array}[t]{*{25}{r}}
\big( x^4\pb&-\pb7 x^3\pb&-\pb5 x\pb&+\pb35\big)&:&\big( x-7\big) =  x^3-5\\
-\big( x^4\pb&-\pb7 x^3\big) \\\cline{1-2}&&
-\pb5 x\pb&+\pb35\pb\\&&
-\big(-\pb5 x\pb&+\pb35\big) \\\cline{3-4}&&&0\pb
\end{array}$
\item $\arraycolsep=1.4pt\newcommand{\pb}{\phantom{\big)}}
\begin{array}[t]{*{25}{r}}
\big( x^5\pb&-\pb2 x^3\pb&-\pb8 x\big)&:&\big( x^2+2\big) =  x^3-4 x\\
-\big( x^5\pb&+\pb2 x^3\big) \\\cline{1-2}&
-\pb4 x^3\pb&-\pb8 x\pb\\&
-\big(-\pb4 x^3\pb&-\pb8 x\big) \\\cline{2-3}&&0\pb
\end{array}$
\newpage
\item $\arraycolsep=1.4pt\newcommand{\pb}{\phantom{\big)}}
\begin{array}[t]{*{25}{r}}
\big( x^7\pb&&+\pb2 x^4\pb&&+\pb2 x^2\pb&-\pb x\pb&+\pb2\big)&:&\big( x^4+ x^2+1\big) =  x^3- x+2\\
-\big( x^7\pb&+\pb x^5\pb&&+\pb x^3\big) \\\cline{1-4}&
-\pb x^5\pb&+\pb2 x^4\pb&-\pb x^3\pb&+\pb2 x^2\pb&-\pb x\pb\\&
-\big(-\pb x^5\pb&&-\pb x^3\pb&&-\pb x\big) \\\cline{2-6}&&
2 x^4\pb&&+\pb2 x^2\pb&&+\pb2\pb\\&&
-\big(2 x^4\pb&&+\pb2 x^2\pb&&+\pb2\big) \\\cline{3-7}&&&&&&0\pb
\end{array}$
\item $\arraycolsep=1.4pt\newcommand{\pb}{\phantom{\big)}}
\begin{array}[t]{*{25}{r}}
\big(6 x^6\pb&+\pb6 x^4\pb&+\pb4 x^3\pb&-\pb10 x^2\pb&+\pb4 x\pb&-\pb10\big)&:&\big(2 x^2+2\big) = 3 x^4+2 x-5\\
-\big(6 x^6\pb&+\pb6 x^4\big) \\\cline{1-2}&&
4 x^3\pb&-\pb10 x^2\pb&+\pb4 x\pb\\&&
-\big(4 x^3\pb&&+\pb4 x\big) \\\cline{3-5}&&&
-\pb10 x^2\pb&&-\pb10\pb\\&&&
-\big(-\pb10 x^2\pb&&-\pb10\big) \\\cline{4-6}&&&&&0\pb
\end{array}$
\item $\arraycolsep=1.4pt\newcommand{\pb}{\phantom{\big)}}
\begin{array}[t]{*{25}{r}}
\big( x^9\pb&&&&&&&&&-\pb512\big)&:&\big( x-2\big) = \\
-\big( x^9\pb&-\pb2 x^8\big) \\\cline{1-2}&
2 x^8\pb&&&&&&&&-\pb512\pb\\&
-\big(2 x^8\pb&-\pb4 x^7\big) \\\cline{2-3}&&
4 x^7\pb&&&&&&&-\pb512\pb\\&&
-\big(4 x^7\pb&-\pb8 x^6\big) \\\cline{3-4}&&&
8 x^6\pb&&&&&&-\pb512\pb\\&&&
-\big(8 x^6\pb&-\pb16 x^5\big) \\\cline{4-5}&&&&
16 x^5\pb&&&&&-\pb512\pb\\&&&&
-\big(16 x^5\pb&-\pb32 x^4\big) \\\cline{5-6}&&&&&
32 x^4\pb&&&&-\pb512\pb\\&&&&&
-\big(32 x^4\pb&-\pb64 x^3\big) \\\cline{6-7}&&&&&&
64 x^3\pb&&&-\pb512\pb\\&&&&&&
-\big(64 x^3\pb&-\pb128 x^2\big) \\\cline{7-8}&&&&&&&
128 x^2\pb&&-\pb512\pb\\&&&&&&&
-\big(128 x^2\pb&-\pb256 x\big) \\\cline{8-9}&&&&&&&&
256 x\pb&-\pb512\pb\\&&&&&&&&
-\big(256 x\pb&-\pb512\big) \\\cline{9-10}\multicolumn{7}{l}{=x^8+2 x^7+4 x^6+8 x^5+16 x^4+32 x^3+64 x^2+128 x+256}&&&0\pb
\end{array}$
\end{aufgaben}
\subsection{Polynomdivision mit Rest} \label{rest}
\chead{Lösungen -- Polynomdivision mit Rest}

\begin{aufgaben}
\item $\arraycolsep=1.4pt\newcommand{\pb}{\phantom{\big)}}
\begin{array}[t]{*{25}{r}}
\big( x^2\pb&-\pb4 x\pb&+\pb3\big)&:&\big( x+5\big) =  x-9+\dfrac{48}{ x+5}\\
-\big( x^2\pb&+\pb5 x\big) \\\cline{1-2}&
-\pb9 x\pb&+\pb3\pb\\&
-\big(-\pb9 x\pb&-\pb45\big) \\\cline{2-3}&&
48\pb
\end{array}$
\item $\arraycolsep=1.4pt\newcommand{\pb}{\phantom{\big)}}
\begin{array}[t]{*{25}{r}}
\big( x^2\pb&-\pb7 x\pb&-\pb1\big)&:&\big( x-2\big) =  x-5+\dfrac{-11}{ x-2} = x-5-\dfrac{11}{x-2}\\
-\big( x^2\pb&-\pb2 x\big) \\\cline{1-2}&
-\pb5 x\pb&-\pb1\pb\\&
-\big(-\pb5 x\pb&+\pb10\big) \\\cline{2-3}&&
-\pb11\pb
\end{array}$
\item $\arraycolsep=1.4pt\newcommand{\pb}{\phantom{\big)}}
\begin{array}[t]{*{25}{r}}
\big( x^2\pb&+\pb10 x\pb&+\pb25\big)&:&\big( x-5\big) =  x+15+\dfrac{100}{ x-5}\\
-\big( x^2\pb&-\pb5 x\big) \\\cline{1-2}&
15 x\pb&+\pb25\pb\\&
-\big(15 x\pb&-\pb75\big) \\\cline{2-3}&&
100\pb
\end{array}$
\item $\arraycolsep=1.4pt\newcommand{\pb}{\phantom{\big)}}
\begin{array}[t]{*{25}{r}}
\big( x^4\pb&+\pb2 x^3\pb&+\pb6 x^2\pb&&+\pb1\big)&:&\big( x+3\big) =  x^3- x^2+9 x-27+\dfrac{82}{ x+3}\\
-\big( x^4\pb&+\pb3 x^3\big) \\\cline{1-2}&
-\pb x^3\pb&+\pb6 x^2\pb&\\&
-\big(-\pb x^3\pb&-\pb3 x^2\big) \\\cline{2-3}&&
9 x^2\pb&&+\pb1\pb\\&&
-\big(9 x^2\pb&+\pb27 x\big) \\\cline{3-4}&&&
-\pb27 x\pb&+\pb1\pb\\&&&
-\big(-\pb27 x\pb&-\pb81\big) \\\cline{4-5}&&&&
82\pb
\end{array}$
\item $\arraycolsep=1.4pt\newcommand{\pb}{\phantom{\big)}}
\begin{array}[t]{*{25}{r}}
\big( x^3\pb&-\pb4 x^2\pb&+\pb2 x\pb&-\pb5\big)&:&\big( x^2+3\big) =  x-4+\dfrac{- x+7}{ x^2+3} = x-4-\dfrac{x-7}{x^2+3} \\
-\big( x^3\pb&&+\pb3 x\big) \\\cline{1-3}&
-\pb4 x^2\pb&-\pb x\pb& -\pb5\pb\\&
-\big(-\pb4 x^2\pb&&-\pb12\big) \\\cline{2-4}&&
-\pb x\pb&+\pb7\pb
\end{array}$
\item $\arraycolsep=1.4pt\newcommand{\pb}{\phantom{\big)}}
\begin{array}[t]{*{25}{r}}
\big( x^5\pb&-\pb x^4\pb&+\pb x^3\pb&-\pb x^2\pb&+\pb x\pb&-\pb1\big)&:&\big( x^3- x^2+1\big) =  x^2+1+\dfrac{- x^2+ x-2}{ x^3- x^2+1} \\
-\big( x^5\pb&-\pb x^4\pb&&+\pb x^2\big) &&&&\hfill\qquad\qquad\qquad= x^2+1 - \dfrac{x^2-x+2}{x^3-x^2+1}\hfill\textcolor{white}{.} \\\cline{1-4}&&
 x^3\pb&-\pb2 x^2\pb&+\pb x\pb&-\pb1\pb\\&&
-\big( x^3\pb&-\pb x^2\pb&&+\pb1\big) \\\cline{3-6}&&&
-\pb x^2\pb&+\pb x\pb&-\pb2\pb
\end{array}$
\end{aufgaben}
\subsection{Polynomdivision mit Fehlern} \label{fehler}
\chead{Lösungen -- Polynomdivision mit Fehlern}

Das fehlerhafte Rechenergebnis ist jeweils eingerahmt.

\begin{aufgaben}
\item $\arraycolsep=1.4pt\newcommand{\pb}{\phantom{\big)}}
\begin{array}[t]{*{25}{r}}
\big( x^2\pb&-\pb2 x\pb&-\pb3\big)&:&\big( x-3\big) =  x-5 + \dfrac{-18}{x-3}\\
-\big( x^2\pb&-\pb3 x\big) \\\cline{1-2}&
 \fbox{$-5x$}\pb&-\pb3\pb\\&
-\big( -5x\pb&+\pb15\big) \\\cline{2-3}
&& -18\pb
\end{array}$
\vfill
Korrekt: $\big( x^2-2 x-3\big):\big( x-3\big) =  x+1$
\vfill
\item $\arraycolsep=1.4pt\newcommand{\pb}{\phantom{\big)}}
\begin{array}[t]{*{25}{r}}
\big( x^2\pb&+\pb5 x\pb&+\pb6\big)&:&\big( x+2\big) =  x+7+ \dfrac{20}{x+2}\\
-\big( x^2\pb&+\pb2 x\big) \\\cline{1-2}&
\fbox{$7 x$}\pb&+\pb6\pb\\&
-\big(7 x\pb&+\pb14\big) \\\cline{2-3}&&
	\fbox{$20$}\pb
\end{array}$
\vfill
Korrekt: $\big( x^2+5 x+6\big):\big( x+2\big) =  x+3$
\vfill
\item $\arraycolsep=1.4pt\newcommand{\pb}{\phantom{\big)}}
\begin{array}[t]{*{25}{r}}
\big( x^3\pb&-\pb10 x^2\pb&+\pb17 x\pb&-\pb8\big)&:&\big( x-8\big) =  x^2+2 x+1\\
-\big( x^3\pb&-\pb8 x^2\big) \\\cline{1-2}&
\pb\fbox{$2 x^2$}\pb&+\pb17 x\pb&-\pb8\pb\\&
-\big(\pb2 x^2\pb&-\pb16 x\big) \\\cline{2-3}&&
 \fbox{$x$}\pb&-\pb8\pb\\&&
-\big( \pb x\pb&-\pb8\big) \\\cline{3-4} &&& 0\pb
\end{array}$
\vfill
Korrekt: $\big( x^3-10 x^2+17 x-8\big):\big( x-8\big) =  x^2-2 x+1$
\vfill
\item $\arraycolsep=1.4pt\newcommand{\pb}{\phantom{\big)}}
\begin{array}[t]{*{25}{r}}
\big( x^3\pb&-\pb6 x^2\pb&+\pb4 x\pb&+\pb8\big)&:&\big( x-2\big) = \fbox{$x$}-\fbox{$6$} + \dfrac{6x-4}{x-2}\\
-\big( \fbox{$x^3$}\pb&&-\pb2 x\big) \\\cline{1-3}&
-\pb6 x^2\pb&+\pb6 x\pb&+\pb8\\&
-\big(\fbox{$-\pb6 x^2$}\pb&&+\pb12 \big) \\\cline{2-4}&&
\pb6 x\pb&-\pb4\pb
\end{array}$
\vfill
Korrekt: $\big( x^3-6 x^2+4 x+8\big):\big( x-2\big) =  x^2-4 x-4$
\vfill
\item $\arraycolsep=1.4pt\newcommand{\pb}{\phantom{\big)}}
\begin{array}[t]{*{25}{r}}
\big( x^4\pb&+\pb5 x^3\pb&-\pb52 x^2\pb&-\pb356 x\pb&-\pb480\big)&:&\big( x+5\big) =  x^3\\
-\big( x^4\pb&+\pb5 x^3\big) \\\cline{1-2}& 0\pb &\fbox{$-\pb52 x^2\pb$}&\fbox{$-\pb356 x\pb$}&\fbox{$-\pb480$}
\end{array}$
\vfill
Korrekt: $\big( x^4+5 x^3-52 x^2-356 x-480\big):\big( x+5\big) =  x^3-52 x-96$
\vfill
\end{aufgaben}

\subsection{Weiterführende Aufgaben}\label{weiter}
\chead{Lösungen -- Weiterführende Aufgaben}

\begin{aufgaben}
\item  Die verbleibende Gleichung $0= x^2+4$ hat keine weitere reelle Lösung. Das bedeutet, dass die Gleichung $0= x^3 -3 x^2 +4x -12$ nur eine reelle Lösung besitzt.

\item Der letzte Koeffizient (96) von $f$ besitzt folgende Teiler: $1, 2, 3, 4, 6, 8, 12, 16, 24, 32, 48$ und $96$. Jeweils als Positive und als negative Zahl.
\begin{align*}
f(1) &= 1^5-4\cdot 1^4-15\cdot 1^3 + 50\cdot 1^2+64\cdot 1-96 = 0
\end{align*}
Die erste Nullstelle der Funktion lautet $x_1=1$.

$\arraycolsep=1.4pt\newcommand{\pb}{\phantom{\big)}}
\begin{array}[t]{*{25}{r}}
\big( x^5\pb&-\pb4 x^4\pb&-\pb15 x^3\pb&+\pb50 x^2\pb&+\pb64 x\pb&-\pb96\big)&:&\big( x-1\big) =  x^4-3 x^3-18 x^2+32 x+96 = g(x)\\
-\big( x^5\pb&-\pb x^4\big) \\\cline{1-2}&
-\pb3 x^4\pb&-\pb15 x^3\pb&\\&
-\big(-\pb3 x^4\pb&+\pb3 x^3\big) \\\cline{2-3}&&
-\pb18 x^3\pb&+\pb50 x^2\pb&\\&&
-\big(-\pb18 x^3\pb&+\pb18 x^2\big) \\\cline{3-4}&&&
32 x^2\pb&+\pb64 x\pb&\\&&&
-\big(32 x^2\pb&-\pb32 x\big) \\\cline{4-5}&&&&
96 x\pb&-\pb96\pb\\&&&&
-\big(96 x\pb&-\pb96\big) \\\cline{5-6}&&&&&0\pb
\end{array}$

Die Teilermenge des letzten Koeffizienten der Funktion $g$ bleibt gleich.
\begin{align*}
g(1) &= 1^4-3\cdot 1^3-18\cdot 1^2+32\cdot 1+96 = 108 \\
g(-1) &=(-1)^4-3\cdot (-1)^3-18\cdot (-1)^2+32\cdot (-1)+96 = 50 \\
g(2) &= 2^4-3\cdot 2^3-18\cdot 2^2+32\cdot 2+96 = 80 \\
g(-2) &= (-2)^4-3\cdot (-2)^3-18\cdot (-2)^2+32\cdot (-2)+96 = 0 \\
\end{align*}
Die zweite Nullstelle der Funktion lautet $x_2=-2$.

$\arraycolsep=1.4pt\newcommand{\pb}{\phantom{\big)}}
\begin{array}[t]{*{25}{r}}
\big( x^4\pb&-\pb3 x^3\pb&-\pb18 x^2\pb&+\pb32 x\pb&+\pb96\big)&:&\big( x+2\big) =  x^3-5 x^2-8 x+48 = h(x)\\
-\big( x^4\pb&+\pb2 x^3\big) \\\cline{1-2}&
-\pb5 x^3\pb&-\pb18 x^2\pb&\\&
-\big(-\pb5 x^3\pb&-\pb10 x^2\big) \\\cline{2-3}&&
-\pb8 x^2\pb&+\pb32 x\pb&\\&&
-\big(-\pb8 x^2\pb&-\pb16 x\big) \\\cline{3-4}&&&
48 x\pb&+\pb96\pb\\&&&
-\big(48 x\pb&+\pb96\big) \\\cline{4-5}&&&&0\pb
\end{array}$

Die Teilermenge des letzten Koeffizienten von $h$ lautet: $1, 2, 3, 4, 6, 8, 12, 16, 24$ und 48. Da $1, -1$ und 2 keine Nullstelle von $g$ waren, können sie auch keine Nullstelle von $h$ sein.

\begin{align*}
h(-2) &= (-2)^3-5\cdot(-2)^2-8\cdot(-2)+48=36 \\
h(3) &= 3^3-5\cdot3^2-8\cdot3+48 = 6 \\
h(-3)&= (-3)^3-5\cdot(-3)^2-8\cdot(-3)+48=0
\end{align*}
Die zweite Nullstelle der Funktion lautet $x_3=-3$.

$\arraycolsep=1.4pt\newcommand{\pb}{\phantom{\big)}}
\begin{array}[t]{*{25}{r}}
\big( x^3\pb&-\pb5 x^2\pb&-\pb8 x\pb&+\pb48\big)&:&\big( x+3\big) =  x^2-8 x+16 =i(x)\\
-\big( x^3\pb&+\pb3 x^2\big) \\\cline{1-2}&
-\pb8 x^2\pb&-\pb8 x\pb&\\&
-\big(-\pb8 x^2\pb&-\pb24 x\big) \\\cline{2-3}&&
16 x\pb&+\pb48\pb\\&&
-\big(16 x\pb&+\pb48\big) \\\cline{3-4}&&&0\pb
\end{array}$

Die Nullstellen, der verbleibenden quadratischen Funktion, können mittels der Lösungsformel bestimmt werden.
\begin{align*}
x_4&=4 & x_5&=4
\end{align*}
Linearfaktorzerlegung: $y=f(x)=(x-1)(x+2)(x+3)(x-4)(x-4)=(x-1)(x+2)(x+3)(x-4)^2$

\item Im linken Beispiel wird der zu subtrahierende Teil in Klammern geschrieben. Das rechte Beispiel kommt hingegen ohne Klammern aus. Der allgemeine Aufbau und das Ergebnis sind in beiden Varianten gleich.

Mathematisch korrekt sind beide Varianten, da in der rechten Variante lediglich die Klammer aufgelöst wurde, bevor die Subtraktion (oder hier dann ggf. auch Addition) durchgeführt wurden:
\begin{align*}
-\big( x^2-7 x\big) &= -x^2+7x
\end{align*}

\item Anhand des ersten Beispiels werden fünf Varianten exemplarisch dargestellt:
\begin{enumerate}[aa)]
\item $(x^2+6x+8):(x+a) = x+4$
\begin{enumerate}[i:]
\item Vergleich der jeweils letzten Glieder:

Es muss gelten: $4\cdot a = 8 \rightarrow \underline{a = 2}$
\item Lösen der quadratischen Gleichung:

Da das Ergebnis keinen Rest besitzt, muss $x+4$ ein Linearfaktor von der Gleichung  $0=x^2+6x+8$ sein und $-4$ eine Lösung der Gleichung.

Mithilfe der Lösungsformel für quadratische Gleichungen kann
\begin{align*}
x_1&=-2 & x_2&=-4
\end{align*}
bestimmt werden. Die Lösung $x_2=-4$ war bereits bekannt, sodass die erste Lösung $x_1=-2$, als Linearfaktor $x+2$ genutzt wird. $\rightarrow \underline{a=2}$
\item Multiplikation der beiden Linearfaktoren:

\begin{align*}
(x+a)(x+4) &= x^2+(4+a)x+4a
\end{align*}
Im Vergleich mit dem Ausgangspolynom müssen die Koeffizienten übereinstimmen:
\begin{align*}
x^2+(4+a)x+4a&=x^2+6x+8
\end{align*}
Es gilt demnach:
\begin{align*}
4+a&=6 \rightarrow \underline{a=2} \\
4a&=8 \rightarrow \underline{a=2}
\end{align*}

\item Polynomdivision mit Parameter rückwärts:

$\arraycolsep=1.4pt\newcommand{\pb}{\phantom{\big)}}
\begin{array}[t]{*{25}{r}}
\big( x^2\pb&+\pb6 x\pb&+\pb8\big)&:&\big( x+a\big) =  x+4\\
-\big( x^2\pb&+\pb a x\big) \\\cline{1-2}&
(6-a) x\pb&+\pb8\pb\\&
\end{array}$

Wenn nun $(6-a)x$ durch $x$ geteilt wird, muss das Ergebnis 4 lauten:
\begin{align*}
(6-a)x:x&=4 \\
6-a&=4 \rightarrow \underline{a = 2}
\end{align*}
\item Polynomdivision mit Rest:

$\arraycolsep=1.4pt\newcommand{\pb}{\phantom{\big)}}
\begin{array}[t]{*{25}{r}}
\big( x^2\pb&+\pb6 x\pb&+\pb8\big)&:&\big( x+a\big) =x + (6-a) + \dfrac{8-a(6-a)}{x+a} \\
-\big( x^2\pb&+\pb a x\big) \\\cline{1-2}&
(6-a) x\pb&+\pb8\pb\\&
-\big((6-a) x\pb&+\pb a(6-a)\big) \\\cline{2-3}&\multicolumn{2}{r}{8-a(6-a)}\pb
\end{array}$

Da die Ausgangsaufgabe keinen Rest hat, muss dessen Zähler gleich null sein:
\begin{align*}
0&=8-a(6-a) =8-6a+a^2\\
0&=a^2-6a+8
\end{align*}
Mithilfe der Lösungsformel für quadratische Funktionen kann $a_1 = 2$ und $a_2=4$ berechnet werden.

Diese werden nun in das Ergebnis der Polynomdivision eingesetzt und mit der Aufgabe verglichen:
\begin{align*}
a&=2 & a&=4 \\
x+(6-a)&=x+4 & x+(6-a)&=x+4 \\
x+(6-2)&=x+4 & x+(6-4)&=x+4 \\
x+4&=x+4\quad \checkmark & x+2&\neq x+4 \quad \lightning \rightarrow \underline{a=2}
\end{align*}
\end{enumerate}

\item $(x^2-3x-10):(x+a)=x-5 \rightarrow a = 2$
\item $(x^2-4x+16):(x+a)=x-4 \rightarrow a = -4$
\item $(x^3 + 8 x^2 - 69 x - 252):(x+a)=x^2-4x-21 \rightarrow a = 12$
\end{enumerate}
\newpage
\item Beschreibung des Verfahrens mit Division -- Multiplikation -- Subtraktion.

Beschreibung des Verfahrens mit Rest und ggf. fehlenden Gliedern.

\item (In Satzform) Die Folgende Auflistung erhebt keinen Anspruch auf Vollständigkeit.

\begin{itemize}
\item Allgemeine Form:
	\begin{itemize}
	\item[$+$] Vorteile
		\begin{itemize}
		\item Polynomgrad ist einfach ablesbar
		\item Schreibweise meist kompakter
		\item Funktionsgleichung durch mehrere Punkte kann relativ einfach bestimmt werden (Lineares Gleichungsystem)
		\end{itemize}
	\item[$-$] Nachteile
		\begin{itemize}
		\item Anzahl der Nullstellen ist nicht ablesbar
		\item Nullstellen sind teilweise nur schwer zu ermitteln
		\item Umwandlung in Linearfaktorzerlegung kompliziert
		\end{itemize}
	\end{itemize}
\item Linearfaktorzerlegung:
	\begin{itemize}
	\item[$+$] Vorteile
		\begin{itemize}
		\item Anzahl der Nullstellen ist ablesbar
		\item Mehrfachnullstellen sind sofort ersichtlich
		\item Nullstellen sind leicht ablesbar
		\item Umwandlung in Allgemeine Form relativ einfach
		\end{itemize}
	\item[$-$] Nachteile
		\begin{itemize}
		\item Funktionsgleichung durch mehrere Punkte ist nur schwer ermittelbar (Nichtlineares Gleichungssystem)
		\item Schreibweise meist umfangreicher
		\item Polynomgrad ist nicht direkt ablesbar
		\end{itemize}
	\end{itemize}
\end{itemize}
\item Überführung der Schreibweisen:
\begin{align*}
y=f(x)&=(((3x-9)x-24)x+36)x+48 \\
y=f(x)&=((3x^2-9x-24)x+36)x+48 \\
y=f(x)&=(3x^3-9x^2-24x+36)x+48 \\
y=f(x)&=3x^4-9x^3-24x^2+36x+48
\end{align*}

Durch die fehlenden Potenzen ist die Berechnung von Funktionswerten weniger kompliziert und damit schneller, einfacher und effizienter.
\begin{itemize}
\item Einfachere Berechnung von Funktionswerten
\item Effizientere Programmierung in der Informatik
\end{itemize}
\end{aufgaben}


\end{document}
