\section{Weiterführende Aufgaben}\label{wech}

Lösung auf Seite \pageref{weiter}.

\begin{aufgaben}
\item Beim Bestimmen der Lösungen der Gleichung $0= x^3 -3 x^2 +4x -12$ wurde $x_1=3$ als Lösung gefunden. Nach Abspaltung des Linearfaktors $(x-3)$ blieb $0= x^2+4$ übrig. 

Erklären Sie, was dieses Ergebnis für die Anzahl der Lösungen bedeutet!
\vspace{4cm}

\item Berechnen Sie alle Nullstellen der Funktion $y=f(x)=x^5 - 4 x^4 - 15 x^3 + 50 x^2 + 64 x - 96$ und stellen Sie anschließend deren Linearfaktorzerlegung auf!
\vfill

\cleardoublepage

\item Erläutern Sie Unterschiede und Gemeinsamkeiten zwischen den beiden dargestellten Polynomdivisionen!

\begin{multicols}{2}
\arraycolsep=1.4pt\newcommand{\pb}{\phantom{\big)}}
\begin{array}[t]{*{25}{r}}
\big( x^2\pb&-\pb4 x\pb&-\pb21\big)&:&\big( x-7\big) =  x+3\\
-\big( x^2\pb&-\pb7 x\big) \\\cline{1-2}&
3 x\pb&-\pb21\pb\\&
-\big(3 x\pb&-\pb21\big) \\\cline{2-3}&&0\pb
\end{array}

\columnbreak

\polylongdiv[style=C,div=:]{x^2-4x-21}{x-7}
\end{multicols}
Begründen Sie, warum beide Schreibweisen mathematisch korrekt sind!
\vspace*{5cm}

\item Bestimmen Sie den Parameter $a$ so, dass die folgenden Gleichungen stimmen! Erklären Sie Ihr Vorgehen! Können Sie mehr als eine Vorgehensweise beschreiben?
\begin{enumerate}[aa)]
\item $(x^2+6x+8):(x+a) = x+4$
\vfill
\item $(x^2-3x-10):(x+a)=x-5$
\vfill
\item $(x^2-4x+16):(x+a)=x-4$
\vfill
\item $(x^3 + 8 x^2 - 69 x - 252):(x+a)=x^2-4x-21$
\vfill
\end{enumerate}

\cleardoublepage

\item Schreiben Sie einem/r Freund/in einen Brief in dem Sie ihm/ihr mit eigenen Worten das Verfahren der Polynomdivision erklären!
\vspace*{7cm}
\item Diskutieren Sie Vor- und Nachteile der beiden bekannten Schreibweisen für ganzrationale Funktionen:
\begin{itemize}
\item Allgemeine Form: $y = f(x)=3 x^4 - 9 x^3 - 24 x^2 + 36 x + 48$
\item Linearfaktorzerlegung: $y = f(x)=3(x+1)(x-2)(x+2)(x-4)$
\end{itemize}
\vfill
\item Die nachfolgend dargestellte Funktionsgleichung soll die gleiche Funktion wie in Aufgabe f) beschreiben.
\begin{align*}
y &= f(x)=(((3x-9)x-24)x+36)x+48
\end{align*}
Zeigen Sie, dass diese Schreibweise in die allgemeine Form überführbar ist!
\vspace*{5cm}

Notieren Sie wofür diese Funktionsschreibweise in der Praxis genutzt werden könnte!

\vspace*{3cm}
\end{aufgaben}