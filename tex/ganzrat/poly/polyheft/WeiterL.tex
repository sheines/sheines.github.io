\subsection{Weiterführende Aufgaben}\label{weiter}
\chead{Lösungen -- Weiterführende Aufgaben}

\begin{aufgaben}
\item  Die verbleibende Gleichung $0= x^2+4$ hat keine weitere reelle Lösung. Das bedeutet, dass die Gleichung $0= x^3 -3 x^2 +4x -12$ nur eine reelle Lösung besitzt.

\item Der letzte Koeffizient (96) von $f$ besitzt folgende Teiler: $1, 2, 3, 4, 6, 8, 12, 16, 24, 32, 48$ und $96$. Jeweils als Positive und als negative Zahl.
\begin{align*}
f(1) &= 1^5-4\cdot 1^4-15\cdot 1^3 + 50\cdot 1^2+64\cdot 1-96 = 0
\end{align*}
Die erste Nullstelle der Funktion lautet $x_1=1$.

$\arraycolsep=1.4pt\newcommand{\pb}{\phantom{\big)}}
\begin{array}[t]{*{25}{r}}
\big( x^5\pb&-\pb4 x^4\pb&-\pb15 x^3\pb&+\pb50 x^2\pb&+\pb64 x\pb&-\pb96\big)&:&\big( x-1\big) =  x^4-3 x^3-18 x^2+32 x+96 = g(x)\\
-\big( x^5\pb&-\pb x^4\big) \\\cline{1-2}&
-\pb3 x^4\pb&-\pb15 x^3\pb&\\&
-\big(-\pb3 x^4\pb&+\pb3 x^3\big) \\\cline{2-3}&&
-\pb18 x^3\pb&+\pb50 x^2\pb&\\&&
-\big(-\pb18 x^3\pb&+\pb18 x^2\big) \\\cline{3-4}&&&
32 x^2\pb&+\pb64 x\pb&\\&&&
-\big(32 x^2\pb&-\pb32 x\big) \\\cline{4-5}&&&&
96 x\pb&-\pb96\pb\\&&&&
-\big(96 x\pb&-\pb96\big) \\\cline{5-6}&&&&&0\pb
\end{array}$

Die Teilermenge des letzten Koeffizienten der Funktion $g$ bleibt gleich.
\begin{align*}
g(1) &= 1^4-3\cdot 1^3-18\cdot 1^2+32\cdot 1+96 = 108 \\
g(-1) &=(-1)^4-3\cdot (-1)^3-18\cdot (-1)^2+32\cdot (-1)+96 = 50 \\
g(2) &= 2^4-3\cdot 2^3-18\cdot 2^2+32\cdot 2+96 = 80 \\
g(-2) &= (-2)^4-3\cdot (-2)^3-18\cdot (-2)^2+32\cdot (-2)+96 = 0 \\
\end{align*}
Die zweite Nullstelle der Funktion lautet $x_2=-2$.

$\arraycolsep=1.4pt\newcommand{\pb}{\phantom{\big)}}
\begin{array}[t]{*{25}{r}}
\big( x^4\pb&-\pb3 x^3\pb&-\pb18 x^2\pb&+\pb32 x\pb&+\pb96\big)&:&\big( x+2\big) =  x^3-5 x^2-8 x+48 = h(x)\\
-\big( x^4\pb&+\pb2 x^3\big) \\\cline{1-2}&
-\pb5 x^3\pb&-\pb18 x^2\pb&\\&
-\big(-\pb5 x^3\pb&-\pb10 x^2\big) \\\cline{2-3}&&
-\pb8 x^2\pb&+\pb32 x\pb&\\&&
-\big(-\pb8 x^2\pb&-\pb16 x\big) \\\cline{3-4}&&&
48 x\pb&+\pb96\pb\\&&&
-\big(48 x\pb&+\pb96\big) \\\cline{4-5}&&&&0\pb
\end{array}$

Die Teilermenge des letzten Koeffizienten von $h$ lautet: $1, 2, 3, 4, 6, 8, 12, 16, 24$ und 48. Da $1, -1$ und 2 keine Nullstelle von $g$ waren, können sie auch keine Nullstelle von $h$ sein.

\begin{align*}
h(-2) &= (-2)^3-5\cdot(-2)^2-8\cdot(-2)+48=36 \\
h(3) &= 3^3-5\cdot3^2-8\cdot3+48 = 6 \\
h(-3)&= (-3)^3-5\cdot(-3)^2-8\cdot(-3)+48=0
\end{align*}
Die zweite Nullstelle der Funktion lautet $x_3=-3$.

$\arraycolsep=1.4pt\newcommand{\pb}{\phantom{\big)}}
\begin{array}[t]{*{25}{r}}
\big( x^3\pb&-\pb5 x^2\pb&-\pb8 x\pb&+\pb48\big)&:&\big( x+3\big) =  x^2-8 x+16 =i(x)\\
-\big( x^3\pb&+\pb3 x^2\big) \\\cline{1-2}&
-\pb8 x^2\pb&-\pb8 x\pb&\\&
-\big(-\pb8 x^2\pb&-\pb24 x\big) \\\cline{2-3}&&
16 x\pb&+\pb48\pb\\&&
-\big(16 x\pb&+\pb48\big) \\\cline{3-4}&&&0\pb
\end{array}$

Die Nullstellen, der verbleibenden quadratischen Funktion, können mittels der Lösungsformel bestimmt werden.
\begin{align*}
x_4&=4 & x_5&=4
\end{align*}
Linearfaktorzerlegung: $y=f(x)=(x-1)(x+2)(x+3)(x-4)(x-4)=(x-1)(x+2)(x+3)(x-4)^2$

\item Im linken Beispiel wird der zu subtrahierende Teil in Klammern geschrieben. Das rechte Beispiel kommt hingegen ohne Klammern aus. Der allgemeine Aufbau und das Ergebnis sind in beiden Varianten gleich.

Mathematisch korrekt sind beide Varianten, da in der rechten Variante lediglich die Klammer aufgelöst wurde, bevor die Subtraktion (oder hier dann ggf. auch Addition) durchgeführt wurden:
\begin{align*}
-\big( x^2-7 x\big) &= -x^2+7x
\end{align*}

\item Anhand des ersten Beispiels werden fünf Varianten exemplarisch dargestellt:
\begin{enumerate}[aa)]
\item $(x^2+6x+8):(x+a) = x+4$
\begin{enumerate}[i:]
\item Vergleich der jeweils letzten Glieder:

Es muss gelten: $4\cdot a = 8 \rightarrow \underline{a = 2}$
\item Lösen der quadratischen Gleichung:

Da das Ergebnis keinen Rest besitzt, muss $x+4$ ein Linearfaktor von der Gleichung  $0=x^2+6x+8$ sein und $-4$ eine Lösung der Gleichung.

Mithilfe der Lösungsformel für quadratische Gleichungen kann
\begin{align*}
x_1&=-2 & x_2&=-4
\end{align*}
bestimmt werden. Die Lösung $x_2=-4$ war bereits bekannt, sodass die erste Lösung $x_1=-2$, als Linearfaktor $x+2$ genutzt wird. $\rightarrow \underline{a=2}$
\item Multiplikation der beiden Linearfaktoren:

\begin{align*}
(x+a)(x+4) &= x^2+(4+a)x+4a
\end{align*}
Im Vergleich mit dem Ausgangspolynom müssen die Koeffizienten übereinstimmen:
\begin{align*}
x^2+(4+a)x+4a&=x^2+6x+8
\end{align*}
Es gilt demnach:
\begin{align*}
4+a&=6 \rightarrow \underline{a=2} \\
4a&=8 \rightarrow \underline{a=2}
\end{align*}

\item Polynomdivision mit Parameter rückwärts:

$\arraycolsep=1.4pt\newcommand{\pb}{\phantom{\big)}}
\begin{array}[t]{*{25}{r}}
\big( x^2\pb&+\pb6 x\pb&+\pb8\big)&:&\big( x+a\big) =  x+4\\
-\big( x^2\pb&+\pb a x\big) \\\cline{1-2}&
(6-a) x\pb&+\pb8\pb\\&
\end{array}$

Wenn nun $(6-a)x$ durch $x$ geteilt wird, muss das Ergebnis 4 lauten:
\begin{align*}
(6-a)x:x&=4 \\
6-a&=4 \rightarrow \underline{a = 2}
\end{align*}
\item Polynomdivision mit Rest:

$\arraycolsep=1.4pt\newcommand{\pb}{\phantom{\big)}}
\begin{array}[t]{*{25}{r}}
\big( x^2\pb&+\pb6 x\pb&+\pb8\big)&:&\big( x+a\big) =x + (6-a) + \dfrac{8-a(6-a)}{x+a} \\
-\big( x^2\pb&+\pb a x\big) \\\cline{1-2}&
(6-a) x\pb&+\pb8\pb\\&
-\big((6-a) x\pb&+\pb a(6-a)\big) \\\cline{2-3}&\multicolumn{2}{r}{8-a(6-a)}\pb
\end{array}$

Da die Ausgangsaufgabe keinen Rest hat, muss dessen Zähler gleich null sein:
\begin{align*}
0&=8-a(6-a) =8-6a+a^2\\
0&=a^2-6a+8
\end{align*}
Mithilfe der Lösungsformel für quadratische Funktionen kann $a_1 = 2$ und $a_2=4$ berechnet werden.

Diese werden nun in das Ergebnis der Polynomdivision eingesetzt und mit der Aufgabe verglichen:
\begin{align*}
a&=2 & a&=4 \\
x+(6-a)&=x+4 & x+(6-a)&=x+4 \\
x+(6-2)&=x+4 & x+(6-4)&=x+4 \\
x+4&=x+4\quad \checkmark & x+2&\neq x+4 \quad \lightning \rightarrow \underline{a=2}
\end{align*}
\end{enumerate}

\item $(x^2-3x-10):(x+a)=x-5 \rightarrow a = 2$
\item $(x^2-4x+16):(x+a)=x-4 \rightarrow a = -4$
\item $(x^3 + 8 x^2 - 69 x - 252):(x+a)=x^2-4x-21 \rightarrow a = 12$
\end{enumerate}
\newpage
\item Beschreibung des Verfahrens mit Division -- Multiplikation -- Subtraktion.

Beschreibung des Verfahrens mit Rest und ggf. fehlenden Gliedern.

\item (In Satzform) Die Folgende Auflistung erhebt keinen Anspruch auf Vollständigkeit.

\begin{itemize}
\item Allgemeine Form:
	\begin{itemize}
	\item[$+$] Vorteile
		\begin{itemize}
		\item Polynomgrad ist einfach ablesbar
		\item Schreibweise meist kompakter
		\item Funktionsgleichung durch mehrere Punkte kann relativ einfach bestimmt werden (Lineares Gleichungsystem)
		\end{itemize}
	\item[$-$] Nachteile
		\begin{itemize}
		\item Anzahl der Nullstellen ist nicht ablesbar
		\item Nullstellen sind teilweise nur schwer zu ermitteln
		\item Umwandlung in Linearfaktorzerlegung kompliziert
		\end{itemize}
	\end{itemize}
\item Linearfaktorzerlegung:
	\begin{itemize}
	\item[$+$] Vorteile
		\begin{itemize}
		\item Anzahl der Nullstellen ist ablesbar
		\item Mehrfachnullstellen sind sofort ersichtlich
		\item Nullstellen sind leicht ablesbar
		\item Umwandlung in Allgemeine Form relativ einfach
		\end{itemize}
	\item[$-$] Nachteile
		\begin{itemize}
		\item Funktionsgleichung durch mehrere Punkte ist nur schwer ermittelbar (Nichtlineares Gleichungssystem)
		\item Schreibweise meist umfangreicher
		\item Polynomgrad ist nicht direkt ablesbar
		\end{itemize}
	\end{itemize}
\end{itemize}
\item Überführung der Schreibweisen:
\begin{align*}
y=f(x)&=(((3x-9)x-24)x+36)x+48 \\
y=f(x)&=((3x^2-9x-24)x+36)x+48 \\
y=f(x)&=(3x^3-9x^2-24x+36)x+48 \\
y=f(x)&=3x^4-9x^3-24x^2+36x+48
\end{align*}

Durch die fehlenden Potenzen ist die Berechnung von Funktionswerten weniger kompliziert und damit schneller, einfacher und effizienter.
\begin{itemize}
\item Einfachere Berechnung von Funktionswerten
\item Effizientere Programmierung in der Informatik
\end{itemize}
\end{aufgaben}