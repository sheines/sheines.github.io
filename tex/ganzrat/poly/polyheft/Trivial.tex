\section{Sehr einfache Polynomdivision} \label{sech}

Lösung auf Seite \pageref{trivial}.

Dieses Kapitel sollte nur bearbeitet werden, wenn das Verfahren heute zum ersten Mal angewendet wird.

Lösen Sie die folgenden Quotienten mithilfe der Polynomdivision!

Beispiel:
\begin{center}
$\arraycolsep=1.4pt\newcommand{\pb}{\phantom{\big)}}
\begin{array}{*{25}{r}}
\big( x^2\pb&+\pb5 x\pb&+\pb6\big)&:&\big( x+3\big) =  x+2\\
-\big( x^2\pb&+\pb3 x\big) \\\cline{1-2}&
2 x\pb&+\pb6\pb\\&
-\big(2 x\pb&+\pb6\big) \\\cline{2-3}&& 0\pb
\end{array}$
\end{center}

\begin{aufgaben}
\item $\arraycolsep=1.4pt\newcommand{\pb}{\phantom{\big)}}
\begin{array}[t]{*{25}{r}}
\big( x^2\pb&+\pb2 x\pb&-\pb15\big)&:&\big( x+5\big) = 
\end{array}$
\vfill
\item $\arraycolsep=1.4pt\newcommand{\pb}{\phantom{\big)}}
\begin{array}[t]{*{25}{r}}
\big( x^2\pb&-\pb8 x\pb&+\pb7\big)&:&\big( x-7\big) = 
\end{array}$
\vfill
\item $\arraycolsep=1.4pt\newcommand{\pb}{\phantom{\big)}}
\begin{array}[t]{*{25}{r}}
\big( x^2\pb&+\pb12 x\pb&+\pb27\big)&:&\big( x+9\big) =  
\end{array}$
\vfill
\item $\arraycolsep=1.4pt\newcommand{\pb}{\phantom{\big)}}
\begin{array}[t]{*{25}{r}}
\big(2 x^2\pb&+\pb3 x\pb&-\pb20\big)&:&\big( x+4\big) = 
\end{array}$
\vfill
\item $\arraycolsep=1.4pt\newcommand{\pb}{\phantom{\big)}}
\begin{array}[t]{*{25}{r}}
\big(12 x^2\pb&-\pb18 x\pb&-\pb12\big)&:&\big(4 x+2\big) = 
\end{array}$
\vfill
\end{aufgaben}