\section{Selbsteinschätzung}

Erklärung:
\begin{itemize}
\item[\Ninja] Ich beherrsche diesen Aufgabentyp so sicher, dass ich ihn anderen erklären könnte.
\item[\Smiley] Ich bin sicher bei der Ausführung dieses Aufgabentyps.
\item[\Neutrey] Ich muss diesen Aufgabentyp noch üben.
\item[\Sadey] Ich brauche noch Hilfe bei der Lösung dieses Aufgabentyps
\end{itemize}

Die Seitenzahl gibt an, wo die entsprechenden Übungsaufgaben zu finden sind.

\begin{center}
\begin{tabular}{m{3.1cm}|C{6.2cm}|c|*{4}{|C{0.35cm}}|p{2.9cm}}
Aufgabentyp & Beispielaufgabe & Seite & \Ninja & \Smiley & \Neutrey & \Sadey & Häufigste Fehler \\ \hline \hline
Ich kann Summen addieren, subtrahieren und multiplizieren. & $(x-3)(x+5)=$ & \pageref{grch} &&&& \\ \hline

Ich kann mit den Potenzgesetzen umgehen. & $x^6:x^2=$ &\pageref{grch} &&&& \\ \hline

Ich kann sehr einfache Poly\-nom\-divi\-sionen durchführen. & $(x^2-4x+4):(x-2)=$ &\pageref{sech} &&&& \\ \hline

Ich kann einfache Poly\-nom\-divi\-sionen durchführen. & $(x^3-5x^2+2x-4):(x+3)=$ & \pageref{ech} &&&& \\ \hline

Ich kann komplexe Poly\-nom\-divis\-ionen durchführen. & $(x^3+3x^2-6x+2):(x^2+3x-1)=$ & \pageref{koch} &&&& \\ \hline

Ich kann Poly\-nom\-divi\-sion mit fehlenden Gliedern durchführen. & $(x^7-1):(x-1)=$ & \pageref{fgch} &&&& \\ \hline

Ich kann Poly\-nom\-divi\-sion mit Rest durchführen. & $(x^2+1):(x+5)=$ & \pageref{rech} &&&& \\ \hline

Ich kann Fehler in Poly\-nom\-divi\-sionen finden. & $3\cdot 5 = 12$ & \pageref{fech} &&&& \\ \hline

Ich bin sicher im Umgang mit allen oben genannten Aufgabentypen. & & \pageref{wech} &&&& 
\end{tabular}
\end{center}