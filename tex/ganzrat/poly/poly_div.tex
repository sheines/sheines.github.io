\documentclass[11pt]{standalone}

\usepackage{helvet}

\usepackage{amsmath}

\usepackage{ifthen}
\usepackage{tikz} 
\usepackage{pgfplots}
\usetikzlibrary{shapes.misc}
\usetikzlibrary{arrows,arrows.meta}
\usetikzlibrary{calc,intersections, patterns, math}
\usetikzlibrary{decorations.pathmorphing}
\usetikzlibrary{angles,quotes}
\usetikzlibrary{arrows.meta}
\usetikzlibrary{calc,angles,quotes,babel,shapes}
\usetikzlibrary{decorations}
\usepgfplotslibrary{fillbetween}
\usetikzlibrary{backgrounds}



\definecolor{pfeil}{RGB}{168,167,167}
\definecolor{petrol}{RGB}{0, 118, 136}
\definecolor{darkgoldenrod}{RGB}{184, 134, 11}
\colorlet{petrol-lighter}{petrol!40}
\colorlet{darkgoldenrod-lighter}{darkgoldenrod!40} 

\newcommand{\link}[2]{\tikz[baseline]{\node[anchor=base] (#1) {$#2$}}}

\begin{document}

\tikzstyle{every picture}+=[remember picture]

\begin{align*}
	\arraycolsep=1.4pt\newcommand{\pb}{\,} % \phantom{\big)}
	\begin{array}[t]{*{25}{r}}
		\big( \tikz[baseline]{\node[anchor=base] (a1) {$x^3$}}&-\pb x^2\pb &\tikz[baseline]{\node[anchor=base] (a5) {$-\pb14x$}}\pb&\tikz[baseline]{\node[anchor=base] (a9) {$+\pb24$}}\big)&:&\big( x+4\big) =  \tikz[baseline]{\node[anchor=base] (e1) {$x^2$}}\tikz[baseline]{\node[anchor=base] (e2) {$-5 x$}}\tikz[baseline]{\node[anchor=base] (e3) {$+6$}}\\
		-\big( \tikz[baseline]{\node[anchor=base] (a3) {$x^3$}}\pb&+\pb4 x^2\,\big)\tikz[baseline]{\node[] (a2) {}} \\\cline{1-2}&
		\tikz[baseline]{\node[anchor=base] (a17) {$-\pb5 x^2$}}\tikz[baseline]{\node[] (a4) {}}&\tikz[baseline]{\node[anchor=base] (a6) {$-\pb14x$}} \pb&\\&
		\hspace*{-1.5em}-\big(-\pb5 x^2\pb&-\pb20 x\big)\tikz[baseline]{\node[] (a7) {}} \\\cline{2-3}&&
		\tikz[baseline]{\node[anchor=base] (a11) {$6x\pb$}}\tikz[baseline]{\node[] (a8) {}}&\tikz[baseline]{\node[anchor=base] (a10) {$+\pb24\pb$}}\\&&
		-\big(6 x\pb&+\pb24\big)\tikz[baseline]{\node[] (a12) {}} \\\cline{3-4}&&&
		0\pb\tikz[baseline]{\node[] (a13){}}
	\end{array}
\end{align*}

\begin{tikzpicture}[overlay]
	\draw[-stealth, blue, thick] (a1) to [in=160, out=20] node[above,sloped] {$x^3:x=$} (e1); % 
	% \draw[-stealth, green!50!black, thick] (e1) to [in=0, out=210] node[below,sloped] {$x^2\cdot (x+4)=$} (a2);
	% \draw[-stealth, orange, thick] (a2) to [out=-20, in=20] node[right] {$(x^3-x^2) - (x^3+4x^2)=$} (a4);
	% \draw[-stealth, magenta, thick] (a5) to (a6);
	% \draw[-stealth, blue, thick] (a17) to [in=250, out=-30] node[below,sloped] {$-5x^2:x=$} (e2);
	% \draw[-stealth, green!50!black, thick] (e2) to [in=0, out=210] node[below,sloped] {$-5x\cdot (x+4)=$} (a7);
	% \draw[-stealth, orange, thick] (a7) to [out=-20, in=20] node[right] {$(-5x^2-14x) - (-5x^2-20x)=$} (a8);
	% \draw[-stealth, magenta, thick] (a9) to (a10);
	% \draw[-stealth, blue, thick] (a11) to [in=250, out=-30] node[below,sloped] {$6x:x=$} (e3);
	% \draw[-stealth, green!50!black, thick] (e3) to [in=0, out=270] node[below,sloped] {$6\cdot (x+4)=$} (a12);
	% \draw[-stealth, orange, thick] (a12) to [out=-20, in=20] node[right] {$(6x+24) - (6x+24)=$} (a13);
\end{tikzpicture}
\centering
% \begin{tikzpicture}
% 	\node[blue!30] at (-3,0) {\only<2,3,10,11,18,19>{\color{blue}\bfseries}Division};
% 	\node[green!50!black!30] at (0,0) {\only<5,6,13,14,21,22>{\color{green!50!black}\bfseries}Multiplikation};
% 	\node[orange!30] at (3,0) {\only<7,8,15,16,23,24>{\color{orange}\bfseries}Subtraktion};
% 	\node[magenta!30] at (6,0) {\only<9,17>{\color{magenta}\bfseries}Nachholen};
% \end{tikzpicture}

\end{document}
